\chapter{Related Work}
Wordlength inference is a well-studied topic. The focus is often on fractional bits like the method \citeauthor{src:MinOfFrac} suggests in \cite{src:MinOfFrac} where they try to remove bits from places where they have little value -- this is separate from the goal of this work where correctness is of uttermost importance. There is also work \cite{src:HLSandOpt} by \citeauthor{src:HLSandOpt} which tries to apply C/C++ compiler optimizations for general purpose CPUs to HDL versions of C/C++ -- the paper contains a lot of good discussions about FPGA hardware but the findings could do with a large sample size of programs. A similar work is done by \citeauthor{src:ConFPGA}. Work on actually inferring the wordlength of expressions is more akin to typeinference which is well studied \cite{src:DamasHindleyMilner, src:TypeCheckersBook, src:BiTy}. Though the logic system underlying the most practical parts of wordlength inference may be solvable by a very simple constraint satisfaction algorithm.

There is also some value to be found in the method of software development in this work which is inspired by the Agile development methodology \cite{src:Agile}. Making the developer tooling better is something that also aids in producing better software faster -- that is something we believe is non-controversial and that \cite{src:Agile} agrees with.

This work stands firmly in the space between type checking, optimization, HDLs and methods for software development -- a very weird space indeed.
