\chapter{Discussion}
\label{cha:Discussion}

\section{Results}
The Spade compiler was successfully changed to support ranges and more sophisticated wordlength inference. These changes were not able to be completed to the degree that they could replace the mainline compiler -- but most of the harder technical problems have been solved. 

The table in Figure \ref{figCreativeProcess} shows that the subset of both ''AA'' and ''IA'' results in the most tight ranges and the smallest wordlengths. That a combination of the techniques is superior is hardly surprising since ''IA'' handles multiplication better for some expressions a combination is bound to work better than any of them in isolation. Of special consideration is how ''IA'' handles $0$ a lot better, since ranges bound by 0 are still bound by $0$ even after multiplication, this is not the case for ''AA''. This begs the question if there is a better method that would give even tighter ranges.

The internal changes to the spade-compiler typeinference caused a sea of troubles to appear. If the Spade project want to integrate these changes fully care needs to be taken when implementing and updating language features like memories. There is also a lot of work needed to port all the tests for the Spade language, all of these are fairly manual changes but might require a lot of time. To fully integrate these changes a series of language design decisions would also need to be taken, for example how constants passed to functions should typecheck. Some changes requires more code changes than others.

There was no discernible difference between the PNRd programs with any of the wordlength inference methods. The table in Figure \ref{fig:SpadeCompilations50Table} motivates this -- even though the averages might suggest it. This means there is no real difference in LUTs or memories used when using the wordlength inference in this thesis. It might be claimed that the difference in variance is an improvement, it might more probably be due to random noise from the compiler since PNR has undeterministic behavior. The figures Figure \ref{fig:SMDoutput} and Figure \ref{fig:FIRoutput} support this claim. Though it is worth noting that something might be different in the compilation process since both of these figures show no noise. This might signal an error in the experiments and that the Spade-implementation with the range-based wordlength inference might have a bug which causes incorrect codegeneration, unfortunately there was no access to hardware to verify this implementation on. This is a major limitation of this study.

The implementation of the wordlenght inference relies on both \verb+BigInt+ and \verb+BigRational+. This makes the implementation quite slow, though it gives accurate results.

In the Spade language wordlenght inference is now more flexible, and allows propagation of ranges between functions. That this kind of wordlength information can be sent in more detail between different parts of the program means that developers need fewer assumptions of how the code works. These fewer assumptions result in fewer sign-extensions and truncation operations on integers. The power of AA also makes it beneficial to write larger expressions instead of splitting them up in smaller chunks with truncation operations.

Unfortunately the representation of integer values inside of the Spade compiler is ranges, which hinders some of the AA inference. Since each variable in Spade has a singular range paired with it the information of the sub-expressions -- the information AA uses to function well. For the AA wordlength inference method each variable is opaque and cannot be inspected into, though it is very likely that we knew this information at some point. This also makes it better to write larger expressions to aid the wordlength inference.

\subsection{Limitations in the Spade Compiler}
The implementation offered in this thesis can have quite sporadic error messages. This is due to the typechecker discarding information of where typeinformation came from -- and the typeinference module is left looking at the expressions unless it wishes to implement another typechecker which would have to be in sync with the one already in the language. Changing what the Spade-compiler stores from the typechecking phase would require a fair bit of plumbing but still nothing hugely complicated to implement. For later stages in the compiler it would be preferable to make some changes in the compiler to support:
\begin{itemize}
  \item A list of spans that have been unified to give a type for an expression -- this would make it possible to point to the type signatures from stages that do e.g. constant folding or wordlength inference.
  \item A full list of the constraints and requirements for a type -- currently the compiler tosses these constraints and requirements when they are deemed satisfied, this discards the sources of facts that both the typechecker and wordlength inference module would find helpful.
  \item It would be great if \verb+Requirement+ and \verb+ConstraintExpr+ could be one construction -- it would aid interop and remove some technical debt in the typechecker.
  \item It would also be beneficial if each entity in the Spade program could be compiled as far as possible -- so if typechecking for one function of the program fails the wordlength could still be checked for an entity that is not related.
\end{itemize}

There are also profound profits in the area of usability of the Spade language -- but removing truncation operations from the Spade code can make the code a lot more readable. Since most of the truncations and sign extensions are required by the very simple wordlength inference method present before this thesis this is probably the single largest contribution that has been made. The change in wordlength inference makes the Spade-compiler more flexible and makes it possible to remove a lot of the truncation and extension operations.

\section{Method}
The method of making multiple changes and seeing what works proved beneficial. It was helpful in introducing new developers into the complex codebase that is the Spade compiler. Most of the work done in those sections -- though discarded -- found either a fault in the reasoning for this thesis or a fault in compiler design. The value of these changes is high, according to the authors.

The method does not focus directly on wordlength inference but focuses more on the software development side, which is in stark contrast with the research questions. This might however show more fault in the usage of research questions than this thesis. The authors do believe the code changes that came from this thesis to be among the best possible for the given time. The fumbling evaluation of alternative implementations give a basis for that argument. It is however possible for such an implementation to have been found by thinking very carefully and very hard -- though theoretically sound ideas often crumble in agony when faced with the sledgehammer that is reality. Especially if those theories are drawn from inexperience.

From software development it is well known that an iterative approach to software development often yields the best results in a fixed amount of time. This programmer lore is true for this thesis.

It is also frustrating to leave the compiler in state where most of the work is almost done. As stated earlier in the thesis -- it is possible that some of the changes needed to be done are in fact harder than they appear. But programming work is well known to be hard to plan for. This thesis does however leave a good base to work further on the compiler. It might also be for the better than these other changes are made by someone on the core Spade-compiler team. Though this thesis has been a collaboration with the Spade-compiler team communication is always lossy, and details are bound to be forgotten. Hopefully these changes are well documented enough to be of use and be able to be well integrated into the language.

The way this thesis implemented the 

\section{The Work in a Wider Context}
More sophisticated wordlength inference in Spade unfortunately has little effect on the planet or the current war in Ukraine that is currently raging (at the time of writing). But this contribution need not be discarded as useless. A slight improvement in efficiency for hardware designs can have cascading effects on how cheap it is to produce custom circuits. The cheaper cost can of course lead to an increased production which is what has already happened to goods such as computers and cars. Hardware description languages and FPGAs is used heavily in the weapons industry and thus code created in this thesis might contribute to more raging wars. That said, war is a double edged sword -- and it might save as well as damn -- but it is sure to always destroy.

Hopefully hardware designers will rejoice over these changes.

