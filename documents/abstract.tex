A compiler for a programming languages is a complex programs that can aid software and hardware development a lot. This thesis modifies the Spade hardware description language -- a user friendly programming-language for hardware. When developing hardware the size of all data is very important -- the size for numbers is often called the wordlength. This thesis outlines a method for developing wordlength inference directly into the Spade-language and guessing the size of numbers from the sourcecode of the program alone. This thesis uses multiple smaller implementations and evaluations in an agile-like development process. The wordlength inference is then evaluated by doing multiple place and routes on the same Spade code using \verb|nextpnr|. No measurable difference in hardware-resource usage can be found due to the changes in this thesis. The real value of this work is however in enabling more sophisticated language features in the Spade compiler though the wordlength inference presented in this thesis requires more integration work to function truly well.
