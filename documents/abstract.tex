Compilers, complex programs with the potential to greatly facilitate software and hardware design. This thesis focuses on enhancing the Spade hardware description language, known for its user-friendly approach to hardware design. In the realm of hardware development data size -- for numerical values data size is known as "wordlength" -- plays a critical role for reducing the hardware resources. This study presents an innovative approach that seamlessly integrates wordlength inference directly into the Spade language, enabling the over-estimation of numeric data sizes solely from the program's source code.

The methodology involves iterative development, incorporating various smaller implementations and evaluations, reminiscent of an agile approach. To assess the efficacy of the wordlength inference, multiple place and route operations are performed on identical Spade code using various versions of \verb+nextpnr+. Surprisingly, no discernible impact on hardware resource utilization emerges from the modifications introduced in this thesis.

Nonetheless, the true significance of this endeavor lies in its potential to unlock more advanced language features within the Spade compiler. It is important to note that while the wordlength inference proposed in this thesis shows promise, it necessitates further integration efforts to realize its full potential.
